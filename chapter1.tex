% !TeX spellcheck = <none>
\chapter{Introdução} \label{chap:introdução}
Este capítulo levanta algumas questões a serem respondidas durante a dissertação e fornece uma  visualização dos objetivos e alcance do documento.
 
\section{Enquadramento} \label{sec:enquadramento}
A necessidade de uma maior eficiência nos trabalhos do quotidiano, levou a um desenvolvimento da robótica, tornando os robôs mais autónomos e eficazes. Com o avanço da robótica é possível aumentar o número de horas de trabalho sem perdas de eficiência e desgaste que um ser humano teria, obtendo uma diminuição de custos.

Em Portugal a vinicultura tem vindo a ter um crescimento constante e com ela um aumento da necessidade da robótica. A monitorização das vinhas é essencial para qualidade do produto final , sendo esta, muitas vezes difícil de se realizar com eficácia. Assim , através da robótica essa eficácia pode ser conseguida, realizando trabalhos de maior dificuldade para o Homem a um custo mais reduzido, tal como a monitorização dia e noite, 24h por dia , dos terrenos. 

Desta forma, o projeto \textit{RoMoVi}, desenvolvido pelo centro de Robótica do INESC-TEC, pretende desenvolver um robô, no âmbito da robótica para a agricultura, capaz de podar, monitorizar e fertilizar as encostas de vinhas inclinadas \cite{Mendes2016}. Este projeto tem disponíveis três plataformas : o \textit{AGROB V16}, \textit{AGROB V15}, \textit{AGROB V14}, sendo a presente dissertação desenvolvida no contexto do projeto \textit{RoMoVi} e linha de investigação apelidada de \textit{AGROB} \cite{RN32}.

Neste projeto é necessário a implementação de uma sistema de localização que seja de baixo custo e adequado à aplicação em contexto de vinha de encosta.

A precisão na localização de robôs é fundamental para a sua eficácia e funcionamento. Existem vários tipos de localizações , tais como \textit{Global Position System} (GPS), \textit{Global Navigations Satellite System} (GNSS), \textit{Inertial Navigation System} (INS), Odometria através das rodas, laser/ultrasonico Odometria e Odometria Visual (OV). Contudo, cada tecnologia tem as suas fraquezas. Odometria das rodas é uma tecnologia simples para estimação da posição mas a inclinação do terreno e deslizamento das rodas no pavimento causa erros grandes. Laser/ultrassónico  Odometria utiliza energia acústica para detetar objetos e medir distâncias entre os sensores e os objetos, contudo estes sensores são muito sensiveis ao ruido . INS é ótimo para a acumulação de deslizamento e têm grande precisão, mas em contrapartida é muito dispendiosa a nível monetário e não é a ideal para soluções comerciais. Apesar de a GPS  ser a solução mais comum para localização absoluta por causa da não acumulação de erro, não é útil para este projeto devido ao fraco sinal dos satélites. Tal, deve-se à inclinação dos terrenos e possibilidades de céu nublado. Para além disso, a utilização de GPS este também se torna cara para a precisão em centímetros. \cite{Aqel2016}

Por último, OV é uma tecnologia que envolve um fluxo de imagens adquiridas através de uma câmara e posteriormente analisadas permitindo obter a estimação da localização. É um método barato e com grande precisão, que se enquadra neste projeto. As desvantagens desta localização passam pela necessidade de um bom processamento e a possibilidade de existência de erros causados por objetos dinâmicos e/ou sombras.

\section{Motivação e Objetivos} \label{sec:context}

Com as dificuldades de localização já mencionadas no capitulo anterior, a tecnologia mais indicada é a Odometria Visual (OV). Esta será implementada num raspberry pi com uma câmara com lente olho de peixe para obter maiores ângulos de captura.

Assim, utilizando a câmara com lente de olho de peixe com um ângulo de cerca de 180º é possível capturar um fluxo de imagens com imensa informação. Informação essa que será analisada por um algoritmo implementado num raspberri pi, da qual resultará uma estimativa de localização.

A utilização de visão será sensível às condições de iluminação e reflexões. Com isto, o processamento de imagem terá de ser rigoroso para que o erros sejam mínimos.

Na presente dissertação pretende-se desenvolver um sistema de auxílio à localização que seja de baixo custo e adequado à aplicação em contexto de vinha de encosta. Neste sentido será utilizada a OV na qual resultará um avaliação da precisão do método.
