\chapter{Caracterização do problema}\label{chap:chap4}

Para obter o conhecimento da localização através de OV é necessário calcular a distância entre duas imagens consecutivas. De forma a obter a distância, necessitamos de pontos caraterísticos nas imagens para ser possível a sua deteção na imagem seguinte. Tal, é possível através do conhecimento da formação de uma imagem.

Devido ao meio ambiente em que este projeto está incorporado , a obtenção de imagens com ângulos de cerca 45º (ângulo comum de uma câmara)  limita muito a deteção de pontos caraterísticos. De forma a diminuir esta desvantagem utiliza-se uma lente olho de peixe, com abertura de cerca de 180º para obtenção de  uma maior qualidade e quantidade de pontos caraterísticos. 

Como analisado no capitulo ~\ref{chap:odometria visual}, já existem vários métodos de deteção, associação de pontos caraterísticos e estimação do movimento. Maior parte destes métodos são implementados em câmaras do tipo pinhole , sem distorções. No artigo \cite{6460718}, os autores comparam os vários tipos de detetores e descritores para vários métodos, obtendo tempos de execução para cada método. Em outro artigo, \cite{ImagMatch}, os autores adicionam rotações e distorções às imagens de forma a comparar a eficácia dos métodos. Mas, como analisado em \cite{Zhang2016}, o uso de lentes olho de peixe adiciona muitos benefícios.  Nos artigos, \cite{Forster2014,Forster2017} , os autores implementaram OV em UAVs. Desta forma, torna-se importante a realização do estudo para que ocorra a implementação em  robôs agrícolas com câmaras com lente olho de peixe e sistema monocular.
