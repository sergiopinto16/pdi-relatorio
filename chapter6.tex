% !TeX spellcheck = pt_PT
\chapter{Sistema de Odometria Visual} \label{chap:sist}

Neste capítulo serão apresentados os procedimentos implementados no algoritmo de Odometria Visual e o métodos para a realização dos testes.


\section{Nó VisodoAgro}

Neste subcapítulo é descrito o algoritmo implementado num nó em ROS, que pode ser resumido no diagrama de ação ~\ref{fig:diaguml}

\begin{figure}[h!] %colocar figura a seguir ao texto anterior
	\begin{center}
		\leavevmode		
		\includegraphics[width=1\textwidth]{umldiagram.jpg}
		\caption{Diagrama de ação do nó implementado em ROS.}
		\label{fig:diaguml}
	\end{center}
\end{figure}



No primeiro frame o algoritmo lê a imagem e transforma em escala de cinzas. De seguida, calcula os pontos característicos da imagem. Visto que no primeiro frame não é possível realizar uma correspondência com o frame anterior, devido à inexistência deste, o algoritmo avança para o próximo frame. A partir do segundo frame os passos são sempre iguais. Assim, o algoritmo obtêm o próximo frame e extraí a imagem em escala de cinzas . Seguidamente, caso se utilize o método de \textbf{brute force}, explícito no capítulo ~\ref{subchap:BRUTE}, calcula-se os pontos característicos deste frame que corresponderam aos pontos característicos do anterior. Caso o método seja \textbf{FLANN}, explícito no capítulo ~\ref{subchap:FLANN}, utiliza-se uma pequena região à volta dos pontos característicos do frame anterior como forma de se encontrar o ponto característico do frame atual. 


Obtida a matriz com as devidas correspondências é necessário calcular a matriz Fundamental. Esta, é obtida através de um ciclo em que se escolhe um conjunto de correspondências aleatoriamente de forma a calcular a matriz fundamental. De seguida testa-se a qualidade dessa matriz Fundamental através do número de inliers \footnote{pontos onde está a maior concentração de pontos.}, guardando o conjunto de correspondências com maior número de inliers. No fim calcula a matriz Fundamental com o melhor conjunto de correspondências.

%\textbf{Expressar como a matriz fundamental é calculada.}

\begin{figure}[h!] %colocar figura a seguir ao texto anterior
	\begin{center}
		\leavevmode		
		\includegraphics[width=0.6\textwidth]{epipolar.png}
		\caption{Representação da linha epipolar.}
		\label{fig:equ}
	\end{center}
\end{figure}

Como representado na figura ~\ref{fig:equ} o ponto \textbf{x} corresponde ao ponto no plano de imagem do frame anterior, \textbf{x'} corresponde ao ponto no plano de imagem do frame atual correspondendo os dois ao ponto \textbf{X} no mundo. Desta forma, \textbf{Fx} descreve uma linha (linha epipolar) na qual o ponto correspondente \textbf{x'} na outra imagem deve estar. Isto significa que, para todos os pares de pontos correspondentes: \[ {x}'^{T} F x = 0 \] 
sendo, \[ F =  \left[ \begin{array}{ccc}
f_{11} & f_{12} & f_{13} \\ 
f_{21} & f_{22} & f_{23} \\ 
f_{31} & f_{32} & f_{33} 
\end{array}\right], \] \[ {x}' = \left[ \begin{array}{ccc}
{u}' \\ {v}'\\ 1 
\end{array} \right] e  \thinspace x = \left[ \begin{array}{ccc}
u \\ v \\ 1  \end{array}\right]  . \]

Desenvolvendo, obtemos : \[ u'uf_{11} + u'vf_{12} + u'f_{13} + v'uf_{21} + v'vf_{22} + v'f_{23} + uf_{31} + vf_{32} + f_{33} = 0 \]

\[	\Leftrightarrow (u'u, u'v, u', v'u, v'v, v', u, v, 1) \left( \begin{array}{ccccccccc}
f_{11}\\
f_{12}\\
f_{13}\\
f_{21}\\
f_{22}\\
f_{23}\\
f_{31}\\
f_{32}\\
f_{33}
\end{array} \right) = 0 \] 

Para todos os pontos característicos : \[  \left[ \begin{array}{ccccccccc }
u'_{1}u_{1} & u'_{1}v_{1} & u'_{1} & v'_{1}u_{1} & v'_{1}v_{1} & v'_{1} & u_{1} & v_{1} & 1 \\ 
\vdots  & \vdots  & \vdots  & \vdots  & \vdots  & \vdots  & \vdots  & \vdots  & \vdots \\ 
u'_nu_n & u'_nv_n & u'_n & v'_nu_n & v'_nv_n & v'_n & u_n & v_n & 1
\end{array}\right] \left( \begin{array}{ccccccccc}
f_{11}\\
f_{12}\\
f_{13}\\
f_{21}\\
f_{22}\\
f_{23}\\
f_{31}\\
f_{32}\\
f_{33}
\end{array} \right) = 0 \]  \begin{equation}\label{equ:af=0}
\Leftrightarrow \textbf{AF = 0}. 
\end{equation}

Sendo a matriz \textbf{A} invertível, singular  e o número de pontos característicos superior a 8, a solução é obtida através da decomposição em valores singulares (SVD, do inglês \textit{singular value decomposition}).  \[ F = U D V^{T}.\]

%\textbf{Expressar como é obtida a matriz Essencial através da decomposição de valores singulares}

Conseguida a matriz Fundamental e conhecida a matriz de parâmetros Intrínsecos, \textbf{K}, a matriz Essencial, que representa o plano epipolar na imagem ~\ref{fig:esseciallinemat}, é obtida pela seguinte equação:
\[ E = {K}^{T} F K, \]  sendo \[ E = R \left[\begin{array}{c}
t
\end{array}\right]_{x} \], onde \textbf{R} é a matriz rotação e \textbf{t} é a matriz translação.

\begin{figure}[h!] %colocar figura a seguir ao texto anterior
	\begin{center}
		\leavevmode		
		\includegraphics[width=0.6\textwidth]{epipolar2.jpg}
		\caption{Representação do plano epipolar.\cite{Fraundorfer2012}}
		\label{fig:esseciallinemat}
	\end{center}
\end{figure}

Sendo a matriz Essencial descrita desta forma, é possível obter os valores das matrizes rotação e translação através da decomposição dos valores singulares (\textbf{SVD}). Sendo : \[ E = U diag(1,1,0) V^{T} \] onde as duas soluções possíveis para a matriz rotação, R: \[ R = UWV^T \] \[ R = UW^TV^T \] e a matriz translação, t: \[ t = \pm u_3 \].


Obtidas as matrizes rotação e translação existem 4 possíveis soluções, tal como verificado anteriormente. Desta forma, é necessário escolher a solução certa. Para testar as hipóteses é realizada a triangulação por regressão ortogonal e escolhida a solução mais consistente.

Assim, é possivel verificar que exite uma relação entre os pontos correspondentes entre imagens. Se um ponto desconhecido no mundo, \textbf{X}, representado na frame anterior como \textbf{x} e no frame atual como \textbf{x'}, as coordenadas de \textbf{X} podem ser calculadas. Isto, requer a matriz Intrínseca e a matriz Essencial entre frames.

A relação entre os pontos da imagem, \textbf{x} e \textbf{x'} e o ponto no mundo \textbf{X} com os parâmetros da câmara P é expressa \[ x = P X \] \[ x' = P'X, \] onde P e P' $\epsilon$  $\mathbb{R}^{3x4}$ é a combinação da matriz Intrínseca e Essencial do frame anterior e atual, respetivamente.

Sendo, \[ x =  \left[ \begin{array}{ccc} u \\ v \\ 1 \end{array} \right],  x' =  \left[ \begin{array}{ccc} u' \\ v' \\ 1 \end{array} \right],  X =  \left[ \begin{array}{cccc} X_w \\ Y_w \\ Z_w \\ 1 \end{array} \right], \] \[ E =  \left[ \begin{array}{cccc} r_{11} & r_{12} & r_{13} & t_{x} \\ r_{21} & r_{22} & r_{23} & t_{y} \\ r_{31} & r_{32} & r_{33} & t_{z} \\ \end{array} \right], K =  \left[ \begin{array}{ccc} f_x & 0 & x_c \\ 0 & f_y & y_c \\ 0 & 0 & 1 \\ \end{array} \right]. \]

E considerando $p^{iT}$ as linhas da matriz P e devido à estrutura da matriz Intrínseca, K, X é expresso pela seguinte equação linear :  \begin{equation}\label{equ:svdMatcs} 
\left[ \begin{array}{cccc}
up^{3T} - p^{1T} \\
vp^{3T} - p^{2T} \\
u'p'^{3T} - p'^{1T} \\
v'p'^{3T} - p'^{2T} 
\end{array} \right] X = 0 .
\end{equation}

A equação ~\ref{equ:svdMatcs} é resolvida através da decomposição em valores singulares como explícito anteriormente na equação ~\ref{equ:af=0}.

Desta forma, \[ X = UDV^T. \] 

Onde, X $\epsilon$ $\mathbb{R}^{1xn}$ e n igual ao número de pontos característicos com correspondência.
A matriz X é utilizada na triangulação dos pontos com as combinações das matrizes Rotação e Translação de forma a escolher a combinação com menor erro de triangulação.

\begin{figure}[h!] %colocar figura a seguir ao texto anterior
	\begin{center}
		\leavevmode		
		\includegraphics[width=0.6\textwidth]{homog.png}
		\caption{Representação da transformação homogénea.}
		\label{fig:homog}
	\end{center}
\end{figure}

\vspace{5mm}  %espaço de 5mm para figura ficar antes do seguinte texto

Desta forma, considerando a matriz rotação entre frames \[  R_f = \left[ \begin{array}{cccc}
r_{11} & r_{11} & r_{11} & t_1 \\ 
r_{21} & r_{22} & r_{23} & t_2 \\ 
r_{31} & r_{32} & r_{33} & t_3 
\end{array} \right] = \left(\begin{array}{ccc}
r_1 \\ r_2 \\ r_3
\end{array}\right)\]

e a matriz translação entre frames \[ t_f = \left[ \begin{array}{ccc}
t_x \\ t_y \\ t_z
\end{array}\right]\]

Onde, a matriz rotação global \[ R = R * R_f\]
e a matriz translação global \[ t = t + R * ( t_f * r ) . \]

Originando a matriz concatenação C, \[ C = \left[ R | t \right] = \left[ \begin{array}{cccc}
r_{11} & r_{11} & r_{11} & t_1 \\ 
r_{21} & r_{22} & r_{23} & t_2 \\ 
r_{31} & r_{32} & r_{33} & t_3 \\ 
0 & 0 & 0 & 1
\end{array} \right] \] 






Os tópicos desenvolvidos no nó \textbf{visodoAgro}, foram criadas as seguintes mensagens customizadas : \textit{"msgInfo"}, \textit{"msgTime"} e \textit{"visodom"}, como representado na figura ~\ref{fig:rosgraph}

\begin{figure}[h!] %colocar figura a seguir ao texto anterior
	\begin{center}
		\leavevmode		
		\includegraphics[width=0.9\textwidth]{rosgraph2.png}
		\caption{Representação dos tópicos elaborados no nó visodoAgro.}
		\label{fig:rosgraph}
	\end{center}
\end{figure}

O formato da mensagem \textbf{"msgInfo"} é representado pela figura  ~\ref{fig:msgInfo}

 \begin{figure}[h!] %colocar figura a seguir ao texto anterior
 	\begin{center}
 		\leavevmode		
 		\includegraphics[width=0.4\textwidth]{msgInfo.png}
 		\caption{Conteúdo da mensagem \textit{msgInfo.msg}.}
 		\label{fig:msgInfo}
 	\end{center}
 \end{figure}
 



Esta mensagem é utilizada para publicar todos os dados importantes do algoritmo, para posterior ser realizada uma análise em Matlab e uma representação dos resultados obtidos nesta dissertação. 
Nos campos \textbf{prevImage} e \textbf{currImage} são publicadas as imagens utilizadas para o cálculo das matrizes rotação e transformação nesse instante. Estas matrizes são publicadas no campos \textbf{Rf} e \textbf{tf}, respetivamente. 
O Campo \textbf{C} é usado para publicar a matriz concatenação, enquanto que o campo  \textbf{p\_match} é utilizado para publicar as correspondências dos pontos característicos. Além disso, existem ainda os \textit{inliers}, pontos utilizados para o cálculo das matrizes resultantes, removendo as correspondências que provocavam erros nos resultados.  


A mensagem \textbf{"msgTime"} é utilizada para comparar os tempos de execução, o número de imagens perdidas entre o processamento e quantidade de imagens processadas por diferentes tipos de algoritmos. Esta é representada pelos seguintes campos da figura ~\ref{fig:msgTime}

\begin{figure}[h!] %colocar figura a seguir ao texto anterior
	\begin{center}
		\leavevmode		
		\includegraphics[width=0.4\textwidth]{msgTime.png}
		\caption{Conteúdo da mensagem \textit{msgTime.msg}.}
		\label{fig:msgTime}
	\end{center}
\end{figure}



Por fim, a mensagem \textbf{"visodom"} é utilizada para publicar a localização em relação ao primeiro frame. Esta é constituída pelos seguintes parâmetros da figura ~\ref{fig:visodom}

\begin{figure}[h!] %colocar figura a seguir ao texto anterior
	\begin{center}
		\leavevmode		
		\includegraphics[width=0.4\textwidth]{visodom.png}
		\caption{Constituição da mensagem \textit{visodom.msg}.}
		\label{fig:visodom}
	\end{center}
\end{figure}

As coordenadas x, y e z são publicadas nos parâmetros \textbf{posex}, \textbf{posey}, \textbf{posez}, respetivamente e a orientação alfa,beta e gama nos parâmetros \textbf{orientx}, \textbf{orienty}, \textbf{orientz}, respetivamente. 





\section{Desenvolvimento em Matlab}

O Matlab é um \textit{software} de alto desempenho desenvolvido para o cálculo numérico. Caracteriza-se por implementar uma linguagem própria, que resulta de uma combinação de linguagens como C, Java e Basic.

O código desenvolvido em Matlab teve como principal função o processamento dos dados guardados nos \textit{rosbags}. Estes \textit{rosbags} contêm as mensagens publicadas para cada teste realizado. Foram desenvolvidos \textit{scripts} para filtrar os dados de interesse e gerar dados possíveis de analisar e inferir conclusões.




\section{Tipos de lentes}

Este subcapítulo visa apresentar os tipos de lentes olho de peixe : circular e \textit{full-frame}, sendo que na presente dissertação  foi utilizada a lente olho de peixe \textit{full-frame}.

Os primeiros tipos de lentes olho de peixe desenvolvidos foram circulares. Estas apresentam um ângulo de 180º de visão vertical, horizontal e diagonal e projetam um círculo no plano de imagem.

As lentes olho de peixe \textit{full-frame} ampliam o círculo da imagem para cobrir toda a estrutura retangular, designado por "olho de peixe de moldura completa". Desta forma, as lentes têm um ângulo de visão diagonal de 180º enquanto os ângulos vertical e horizontal de visão são menores. 

A figura ~\ref{fig:circularfullframe} representa as diferenças de plano de imagens obtidos com lente olho de peixe circular,imagem de fundo, olho de peixe \textit{full-frame}, imagem representada no retângulo vermelho, e sem lente, representada no retângulo verde. Desta forma é possível analisar as diferenças entre ângulos de abertura.


A lente é aplicada num ambiente agrícola. Ao se utilizar a lente olho de peixe circular, esta mostra-nos um maior plano de imagem, contudo contém informação desnecessária, como o céu. Assim, torna-se mais vantajoso o uso da lente olho de peixe \textit{full-frame}, uma vez que fornece mais informação central e horizontal do campo de visão.

\begin{figure}[h!] %colocar figura a seguir ao texto anterior
	\begin{center}
		\leavevmode		
		\includegraphics[width=0.8\textwidth]{imageFOV.png}
		\caption{Diferença de ângulos de visão de câmaras com  lentes olho de peixe e sem lentes.}
		\label{fig:circularfullframe}
	\end{center}
\end{figure}


A lente especificamente escolhida é a representada na figura ~\ref{fig:lentfisheye}, com as seguintes especificações: 5 megapíxel, ângulo de abertura de 160 graus (câmaras têm tipicamente 72 graus), resolução 1080p e com suporte para LED infravermelho para visão noturna.


\begin{figure}[!htbp]
	\centering
	\subfloat[\label{fig:lentfisheye}]{
		\includegraphics[width=0.43\textwidth]{lentfisheye}}
	\qquad
	\subfloat[\label{fig:lentfisheyenolent}]{
		\includegraphics[width=0.43\textwidth]{Pi-Camera.jpg}}
	\caption{Câmaras adaptadas para a Raspberry Pi. a) com lente olho de peixe. b) sem lente.}
	\label{fig:lentfisheyeall}
\end{figure}

