\chapter{Sistema de Odometria Visual} \label{chap:sist}

Neste capitulo serão apresentados os procedimentos implementados para a realização dos testes. O capítulo aborda o \textit{software} e \textit{hardware} utilizados no desenvolvimento.

\section{Software}

Neste subcapítulo serão descritas as ferramentas de \textit{software} utilizadas bem como o código produzido.

\subsection{ROS}

ROS (\textit{Robot Operating System}) é uma \textit{framework} para desenvolvimento de \textit{sofware} de robótica. ROS consiste num compêndio de bibliotecas, ferramentas e outros recursos que visa facilitar a criação de comportamentos complexos alcançando uma vasta gama de plataformas robóticas distintas. 

O ROS fornece serviços idêntico a um sistema operativo, incluindo abstração de \textit{hardware}, baixo nivel de controlo, implementação de funcionalidades e mensagens que são transmitidas entre nós. Além disto , é constituido por bibliotecas e ferramentas para obter, criar, gravar e executar código em vários computadores. 

Desta forma, ROS  é caracterizado por :

\begin{itemize}
	\item \textbf{Pacotes} - O \textit{software} desenvolvido em ROS organiza-se em pacotes. Os pacotes são módulos que podem conter nós, bibliotecas independentes, \textit{datasets}, ficherisos de configuração, elementos de \textit{softwares} de terceiros, entre outros elementos que constituem o módulo. Os pacotes constituem a unidade base de compilação.
	\item \textbf{Nós} - Os nós são processos em ROS que executam computação, funcionando como programas. Os nós comunicam entre si através da publicação e subscrição de mensagens, serviços e através do \textit{Parameter Server}. Um sistema robótica em execução utiliza um conjunto de nós que cooperam para o seu funcionamento. A arquitectura distribuída de ROS permite que os nós em execução não necessitem de operar no mesmo equipamento, possibilitando a simbiose de diferentes plataformas comunicando entre si. O nó mestre pertence ao conjunto de nós laçando pelo \textit{roscore}, e tem a função de possibilitar aos nós que se localizem uns aos outros permitindo o estabelecimento de comunicações. 
	\item \textbf{Tópicos} - Os tópicos são os recipientes através dos quais os nós trocam mensagens  entre si. Os tópicos utilizam um modelo de publicação/subscrição que separa a produção de informação do seu consumo. Os nós não têm conhecimento de outros nós que publiquem determinado tópico ou o subscrevam. Um nó apenas necessita de subscrever os dados específicos a um tópico ou de publicar dados num tópico específico. A estrutura permite a existência de múltiplos publicadores e subscritores associados a cada tópico.
	\item \textbf{Serviços} - A unidirecionalidade existente no modelo publicador/subscritor não possibilita interações do tipo "pedido/resposta". Para possibilitar este tipo de interações existem os serviços em ROS. Um serviço é composto por um par de mensagens, uma mensagem definida para o pedido e outra para a resposta. Um nó pode oferecer um serviço ativado por um nó cliente ao submeter a mensagem de pedido.
 
\end{itemize}

A biblioteca de ROS suporta o desenvolvimento ROS em linguagem C++ e Python.

\subsubsection{Desenvolvimento em ROS}

Descrever os nós criados.

\subsection{Desenvolvimento em Matlab}

O Matlab é um \textit{software} de alto desempenho desenvolvido para o cálculo numérico. Caracteriza-se por implementar uma linguagem própria que resulta de uma combinação de linguagens como C, Java e Basic.

O código desenvolvido em Matlab teve como função principal o processamento dos dados guardados nos \textit{rosbags}. Estes \textit{rosbags} contêm as mensagens publicadas para cada teste realizado. Foram desenvolvidos \textit{scripts} para filtrar os dados de interesse e gerar dados possíveis de analisar e inferir conclusões.

\section{Hardware}

Este subcapítulo visa apresentar os componentes de hardware utilizado no desenvolvimento do projeto.

\subsection{Raspberry Pi}

O Raspberry Pi é um computador do tamanho de um cartão de crédito.Todo o hardware é integrado numa única placa. O principal objetivo é promover o ensino em Ciência da Computação básica em escolas mas, devido à sua excelente qualidade / preço é bastante usado em grandes projetos de robótica, programação e até aplicações industriais. 

O modelo utilizado nesta dissertação é o Raspberry Pi 3 model B. Es modelo contem um processador 1.2 GHz 64-bit quad-core ARMv8 CPU, 1 GB de RAM e Bluetooth 4.1. Além disto, este computador é compatível com ROS Kinetic e com vários módulos , quais como a câmara com lente olho de peixe.

\begin{figure}[h!] %colocar figura a seguir ao texto anterior
	\begin{center}
		\leavevmode		
		\includegraphics[width=0.6\textwidth]{raspberry.jpg}
		\caption{Exemplar da Raspberry Pi 3 modelo B.}
		\label{fig:raspberry}
	\end{center}
\end{figure}

\subsection{Câmara com lente olho de peixe}

A lente olho de peixe é uma lente grande angular que produz uma forte distorção visual destinada a criar uma imagem panorâmica ou hemisférica ampla. As lentes olho de peixe alcançam ângulos de visão extremamente amplos. Em vez de produzir imagens com linhas retas de perspetiva, as lentes olho de peixe usam um mapeamento especial (por exemplo: ângulo equissólido), que dá às imagens uma aparência convexa não retilínea.

A lente especificamente escolhida é a representada na figura ~\ref{fig:lentfisheye}, com a seguintes especificações: 5 megapixel , angulo de abertura de 160 graus (câmaras têm tipicamente 72 graus), resolução 1080p e com suporte para LED infravermelho para visão noturna.

\begin{figure}%colocar figura a seguir ao texto anterior
	\begin{center}
		\leavevmode		
		\includegraphics[width=0.6\textwidth]{lentfisheye}
		\caption{Câmara com lente olho de peixe e suporte para visão noturna.}
		\label{fig:lentfisheye}
	\end{center}
\end{figure}
