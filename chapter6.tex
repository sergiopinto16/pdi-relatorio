% !TeX spellcheck = pt_PT
\chapter{Sistema de Odometria Visual} \label{chap:sist}

Neste capitulo serão apresentados os procedimentos implementados para a realização dos testes. O capítulo aborda o \textit{software} e \textit{hardware} utilizados no desenvolvimento.

\section{Software}

Neste subcapítulo serão descritas as ferramentas de \textit{software} utilizadas assim como o código produzido.

\subsection{ROS}

ROS (\textit{Robot Operating System}) é uma \textit{framework} para desenvolvimento de \textit{sofware} de robótica. ROS consiste num compêndio de bibliotecas, ferramentas e outros recursos que visa facilitar a criação de comportamentos complexos alcançando uma vasta gama de plataformas robóticas distintas. 

O ROS fornece serviços idêntico a um sistema operativo, incluindo abstração de \textit{hardware}, baixo nivel de controlo, implementação de funcionalidades e mensagens que são transmitidas entre nós. Além disto , é constituido por bibliotecas e ferramentas para obter, criar, gravar e executar código em vários computadores. 

Desta forma, ROS  é caracterizado por :

\begin{itemize}
	\item \textbf{Pacotes} - O \textit{software} desenvolvido em ROS organiza-se em pacotes. Os pacotes são módulos que podem conter nós, bibliotecas independentes, \textit{datasets}, ficherisos de configuração, elementos de \textit{softwares} de terceiros, entre outros elementos que constituem o módulo. Os pacotes constituem a unidade base de compilação.
	\item \textbf{Nós} - Os nós são processos em ROS que executam computação, funcionando como programas. Os nós comunicam entre si através da publicação e subscrição de mensagens, serviços e através do \textit{Parameter Server}. Um sistema robótica em execução utiliza um conjunto de nós que cooperam para o seu funcionamento. A arquitectura distribuída de ROS permite que os nós em execução não necessitem de operar no mesmo equipamento, possibilitando a simbiose de diferentes plataformas comunicando entre si. O nó mestre pertence ao conjunto de nós laçando pelo \textit{roscore}, e tem a função de possibilitar aos nós que se localizem uns aos outros permitindo o estabelecimento de comunicações. 
	\item \textbf{Tópicos} - Os tópicos são os recipientes através dos quais os nós trocam mensagens  entre si. Os tópicos utilizam um modelo de publicação/subscrição que separa a produção de informação do seu consumo. Os nós não têm conhecimento de outros nós que publiquem determinado tópico ou o subscrevam. Um nó apenas necessita de subscrever os dados específicos a um tópico ou de publicar dados num tópico específico. A estrutura permite a existência de múltiplos publicadores e subscritores associados a cada tópico.
	\item \textbf{Serviços} - A unidirecionalidade existente no modelo publicador/subscritor não possibilita interações do tipo "pedido/resposta". Para possibilitar este tipo de interações existem os serviços em ROS. Um serviço é composto por um par de mensagens, uma mensagem definida para o pedido e outra para a resposta. Um nó pode oferecer um serviço ativado por um nó cliente ao submeter a mensagem de pedido.
 
\end{itemize}

A biblioteca de ROS suporta o desenvolvimento ROS em linguagem C++ e Python.

\subsubsection{Desenvolvimento em ROS}

\subsubsection{Nó VisodoAgro}

	O algoritmo implementado neste nó pode ser resumido ao "seguinte"(mudar para figure) diagrama de atividade.
	
	
	Desta forma, no primeiro frame do algoritmo lê a imagem e transforma em escala de cinzas. De seguida calcula os pontos característicos da imagem. Visto que no primeiro frame não é possível realizar uma correspondência com o frame anterior, devido à inexistência deste, o algortimo avança para o próximo frame. A partir do segundo frame os passos são sempre iguais. Assim, o algoritmo obtêm o próximo frame e extraí a image em escala de cinzas, de seguida, caso se utilize o método de brute force , explicito no capitulo ~\ref{subchap:BRUTE}, é necessário calcular os pontos característicos deste frame que terão uma correspondência aos pontos característicos do anterior. \textit{Caso o método seja FLANN, explicito no capitulo ~\ref{subchap:FLANN}, é procurado o ponto característico no frame atual, numa pequena região à volta do pontos característico do frame anterior, que corresponde.}
	
	Obtido a matriz com as devidas correspondências é necessário calcular a matriz Fundamental. Esta, é obtida através de um loop em que escolhe um conjunto de correspondências aleatoriamente de forma a calcular a matriz fundamental. De seguida testa a qualidade dessa matriz Fundamental através do numero de inliers, guardando o conjunto de correspondências com maior número de inliers. No fim calcula a matriz Fundamental com o melhor conjunto de correspondência.
	
	%\textbf{Expressar como a matriz fundamental é calculada.}
	
	\begin{figure}[h!] %colocar figura a seguir ao texto anterior
		\begin{center}
			\leavevmode		
			\includegraphics[width=0.6\textwidth]{epipolar.png}
			\caption{Representação da linha epipolar.}
			\label{fig:equ}
		\end{center}
	\end{figure}

	Como representado na figura ~\ref{fig:equ} o ponto \textbf{x} corresponde ao ponto no plano de imagem do frame anterior, \textbf{x'} corresponde ao ponto no plano de imagem do frame atual que corresponde os dois ao ponto \textbf{X} no mundo. Desta forma, \textbf{Fx} descreve uma linha (linha epipolar) na qual o ponto correspondente \textbf{x'} na outra imagem deve estar. Isto significa que, para todos os pares de pontos correspondentes : \[ {x}'^{T} F x = 0 \] 
	sendo, \[ F =  \left[ \begin{array}{ccc}
	f_{11} & f_{12} & f_{13} \\ 
	f_{21} & f_{22} & f_{23} \\ 
	f_{31} & f_{32} & f_{33} 
	\end{array}\right], \] \[ {x}' = \left[ \begin{array}{ccc}
	{u}' \\ {v}'\\ 1 
	\end{array} \right] e  \] \[ x = \left[ \begin{array}{ccc}
		u \\ v \\ 1  \end{array}\right]  . \]
		
	Desenvolvendo, obtemos : \[ u'uf_{11} + u'vf_{12} + u'f_{13} + v'uf_{21} + v'vf_{22} + v'f_{23} + uf_{31} + vf_{32} + f_{33} = 0 \]
	
	\[	\Leftrightarrow (u'u , u'v , u' , v'u , v'v , v' , u , v , 1) \left( \begin{array}{ccccccccc}
		f_{11}\\
		f_{12}\\
		f_{13}\\
		f_{21}\\
		f_{22}\\
		f_{23}\\
		f_{31}\\
		f_{32}\\
		f_{33}
		\end{array} \right) = 0 \] 
	
	Para os pontos caracteristicos : \[  \left[ \begin{array}{ccccccccc }
	u'_{1}u_{1} & u'_{1}v_{1} & u'_{1} & v'_{1}u_{1} & v'_{1}v_{1} & v'_{1} & u_{1} & v_{1} & 1 \\ 
	\vdots  & \vdots  & \vdots  & \vdots  & \vdots  & \vdots  & \vdots  & \vdots  & \vdots \\ 
	u'_nu_n & u'_nv_n & u'_n & v'_nu_n & v'_nv_n & v'_n & u_n & v_n & 1
	\end{array}\right] \left( \begin{array}{ccccccccc}
	f_{11}\\
	f_{12}\\
	f_{13}\\
	f_{21}\\
	f_{22}\\
	f_{23}\\
	f_{31}\\
	f_{32}\\
	f_{33}
	\end{array} \right) = 0 \]  \begin{equation}\label{equ:af=0}
	\Leftrightarrow \textbf{AF = 0}. 
	\end{equation}
	
	Sendo a matrix \textbf{A} invertivel, singular  e o numero de pontos caracteristicos superior a 8, a solução é obtida atráves da decomposição em valores singulares (SVD, do inglês \textit{singular value decomposition}).  \[ F = U D V^{T}.\]
	
	%\textbf{Expressar como é obtida a matriz Essencial através da decomposição de valores singulares}
		
	Obtida a matriz Fundamental e conhecida a matriz de parametros íntrinsecos , \textbf{K}, a matriz Essecial, que representa o plano epipolar na imagem ~\ref{fig:esseciallinemat}, é obtida pela seguinte equação:
	\[ E = {K}'^{T} F K, \]  sendo \[ E = R \left[\begin{array}{c}
	t
	\end{array}\right]_{x} \], onde \textbf{R} é a matriz rotação e \textbf{t} é a matriz translação.
	
	\begin{figure}[h!] %colocar figura a seguir ao texto anterior
		\begin{center}
			\leavevmode		
			\includegraphics[width=0.6\textwidth]{epipolar2.jpg}
			\caption{Representação do plano epipolar.}
			\label{fig:esseciallinemat}
		\end{center}
	\end{figure}
	
	Sendo a matriz Essencial descrita desta forma, é possível obter os valores das matrizes rotação e translação através da decomposição dos valores singulares (\textbf{SVD}). Sendo : \[ E = U diag(1,1,0) V^{T} \] onde as duas soluções possíveis para a matriz rotação, R: \[ R = UWV^T \] \[ R = UW^TV^T \] e a matriz translação, t: \[ t = \pm u_3 \].
	
	
	Obtidas as matrizes rotação e translação existem 4 possíveis soluções, tal como verificado anteriormente. Desta forma, é necessário escolher a solução certa. Para testar as hipóteses é realizada a triangulação por regressão ortogonal e escolher a solução mais consistente.
	
	Desta forma, existe uma relação entre os pontos correspondentes entre imagens. Se um ponto desconhecido no mundo, \textbf{X}, representado na frame anterior como \textbf{x} e no frame atual como \textbf{x'}, as coordenas de \textbf{X} podem ser calculadas. Isto, requer a matriz intrinseca e a matriz Essencial entre frames.
	
	A relação entre os pontos da imagem, \textbf{x} e \textbf{x'} e o ponto no mundo \textbf{X} com os parâmetros da câmara P é expressa \[ x = P X \] \[ x' = P'X \], onde P e P' $\epsilon$  $\mathbb{R}^{3x4}$ é a combinação das matrizes intrinseca e Essencial do frame anterior e atual, respetivamente.
	
	Sendo, \[ x =  \left[ \begin{array}{ccc} u \\ v \\ 1 \end{array} \right],  x' =  \left[ \begin{array}{ccc} u' \\ v' \\ 1 \end{array} \right] ,  X =  \left[ \begin{array}{cccc} X_w \\ Y_w \\ Z_w \\ 1 \end{array} \right] \], \[ E =  \left[ \begin{array}{cccc} r_{11} & r_{12} & r_{13} & t_{x} \\ r_{21} & r_{22} & r_{23} & t_{y} \\ r_{31} & r_{32} & r_{33} & t_{z} \\ \end{array} \right] , K =  \left[ \begin{array}{ccc} f_x & 0 & x_c \\ 0 & f_y & y_c \\ 0 & 0 & 1 \\ \end{array} \right] \].
	
	Considerando $p^{iT}$ as linhas da matriz P e devido à estrutura da matriz instrinseca, K, X é expresso pela seguinte equação linear :  \begin{equation}\label{equ:svdMatcs} 
	\left[ \begin{array}{cccc}
	up^{3T} - p^{1T} \\
	vp^{3T} - p^{2T} \\
	u'p'^{3T} - p'^{1T} \\
	v'p'^{3T} - p'^{2T} 
	\end{array} \right] X = 0 .
	\end{equation}
	
	A equação ~\ref{equ:svdMatcs} é resolvidada através da decompoosição em valores singulares como explicito anteriormente na equação ~\ref{equ:af=0}.
	
	Desta forma, \[ X = UDV^T. \] 
	
	Onde X $\epsilon$ $\mathbb{R}^{1xn}$ e n igual ao numero de pontos caracteristicos com correpondência.
	
	Através dos valores de X é possível ajustar a escala de matriz translação. Assim, obtidos os valores da translação em x , y e z e rotação em x, y e z , tx,ty,tz,rx,ry,rz, respetivamente, é calculada a matriz homogénea que depende de todos os pontos caracteristicos.
	--------------
	ajustar texto acima
	inserir imagem transformação 3d
	--------------
	
	Considerando a matriz rotação \[  R = \left[ \begin{array}{cccc}
	r_{11} & r_{11} & r_{11} & t_1 \\ 
	r_{21} & r_{22} & r_{23} & t_2 \\ 
	r_{31} & r_{32} & r_{33} & t_3 
	\end{array} \right] = \left(\begin{array}{ccc}
	r_1 \\ r_2 \\ r_3
	\end{array}\right)\]
	
	e a matriz translação \[ t = \left[ \begin{array}{ccc}
	t_x \\ t_y \\ t_z
	\end{array}\right]\]
	
	estas podem ser combinadas na matriz homogénea, H, \[ H = \left[ \begin{array}{cccc}
	r_{11} & r_{11} & r_{11} & t_1 \\ 
	r_{21} & r_{22} & r_{23} & t_2 \\ 
	r_{31} & r_{32} & r_{33} & t_3 \\ 
	0 & 0 & 0 & 1
	\end{array} \right] \] 
	
	Sendo a matriz rotação ortogonal no espaço 3D, então $R^{T}$ = $R^{-1}$ e o determinante de R é igual a 1, representando $\overrightarrow{r_i}$ = ($r_{i1}$, $r_{i2}$, $r_{i3}$) :\[
	\overrightarrow{r_1}^T \overrightarrow{r_1} = 1,  \overrightarrow{r_2}^T \overrightarrow{r_2} = 1, \overrightarrow{r_3}^T \overrightarrow{r_3} = 1  \]
	\[ \overrightarrow{r_1}^T \overrightarrow{r_2} = 0,  \overrightarrow{r_1}^T \overrightarrow{r_3} = 0, \overrightarrow{r_2}^T \overrightarrow{r_3} = 0 \]
	
	Considerando, \[ r_1 = sin^{-1}(-\frac{r_{22}}{r_{13}}), \] \[ r_2 = sin^{-1}(r_{13}), \] \[ r_3 = sin^{-1}(-\frac{r_{12}}{r_{13}}) \] 
	
	Obtemos a matriz homogenea : \[ H = \left[ \begin{array}{cccc}
	cos(r_2)cos(r_3) & sin(r_1)sin(r_2)cos(r_3) + cos(r_1)sin(r_3) & -cos(r_1)sin(r_2)sin(r_3) + sin(r_1)sin(r_3)  & 0 \\ 
	-cos(r_2)sin(r_3) & -sin(r_1)sin(r_2)sin(r_3) + cos(r_1)cos(r_3)  & cos(r_1)sin(r_2)sin(r_3) + sin(r_1)cos(r_3) & 0 \\ 
	sin(r_2) & -sin(r_1)cos(r_2) & cos(r_1)cos(r_2) & 0 \\ 
	t_x & t_y & t_z & 1
	\end{array} \right] \]
	
	
	Obtida a matriz homogénea entre frames, esta é concatenada com a matriz $C_{k-1}$, considerando k = 0 o instante de tempo do frame atual. Desta forma $C_0$ é a matriz inicial e $C_k$ é a matriz homogénea no instante k em relação ao instante k=0. \[ C_k = C_{k-1} H. \]
	
	
	
	
	-publicação.
	
	\textit{
Devido aos movimentos curtos entre frames causaram maior erro, terem menor precisão no calculo do movimento, é necessário calcular a média do movimentos e aplicar um limite de forma a rejeitar movimentos curtos/lentos. Assim, o frame atual a analisar é rejeitado e analisamos um novo frame de forma a rejeitar movimentos curtos. }
	
	
	
\textbf{-adicionar esquema de nós e explicar}
	
	
	

\subsection{Desenvolvimento em Matlab}

O Matlab é um \textit{software} de alto desempenho desenvolvido para o cálculo numérico. Caracteriza-se por implementar uma linguagem própria que resulta de uma combinação de linguagens como C, Java e Basic.

O código desenvolvido em Matlab teve como função principal o processamento dos dados guardados nos \textit{rosbags}. Estes \textit{rosbags} contêm as mensagens publicadas para cada teste realizado. Foram desenvolvidos \textit{scripts} para filtrar os dados de interesse e gerar dados possíveis de analisar e inferir conclusões.

 \textbf{-explicar o ficheiro .m de forma a criar os gráficos para analise.}

\section{Hardware}

Este subcapítulo visa apresentar os componentes de hardware utilizado no desenvolvimento do projeto.

\subsection{Raspberry Pi}

O Raspberry Pi é um computador do tamanho de um cartão de crédito.Todo o hardware é integrado numa única placa. O principal objetivo é promover o ensino em Ciência da Computação básica em escolas mas, devido à sua excelente qualidade / preço é bastante usado em grandes projetos de robótica, programação e até aplicações industriais. 

O modelo utilizado nesta dissertação é o Raspberry Pi 3 model B. Es modelo contem um processador 1.2 GHz 64-bit quad-core ARMv8 CPU, 1 GB de RAM e Bluetooth 4.1. Além disto, este computador é compatível com ROS Kinetic e com vários módulos , quais como a câmara com lente olho de peixe.

\begin{figure}[h!] %colocar figura a seguir ao texto anterior
	\begin{center}
		\leavevmode		
		\includegraphics[width=0.6\textwidth]{raspberry.jpg}
		\caption{Exemplar da Raspberry Pi 3 modelo B.}
		\label{fig:raspberry}
	\end{center}
\end{figure}

\subsection{Câmara com lente olho de peixe}

A lente olho de peixe é uma lente grande angular que produz uma forte distorção visual destinada a criar uma imagem panorâmica ou hemisférica ampla. As lentes olho de peixe alcançam ângulos de visão extremamente amplos. Em vez de produzir imagens com linhas retas de perspetiva, as lentes olho de peixe usam um mapeamento especial (por exemplo: ângulo equissólido), que dá às imagens uma aparência convexa não retilínea.

\subsubsection{Tipos de lentes}

Existem 2 tipos de lentes olho de peixe, circular e \textit{full-frame}. 

Os primeiros tipos de lentes olho de peixe desenvolvidas foram circulares, lentes que tomaram um hemisfério de 180º projetando um circulo no plano de imagem. Estas, têm 180º de ângulo de visão vertical , horizontal e diagonal. 

As lentes olho de peixe \textit{full-frame} ampliam o círculo da imagem para cobrir toda a estrutura retangular, designado por "olho de peixe de moldura completa". Desta forma, as lentes têm um ângulo de visão diagonal de 180º enquanto os ângulos vertical e horizontal de visão são menores. 

A imagem ~\ref{fig:circularfullframe} representa as diferenças de plano de imagens obtidos com lente olhos de peixe circulares e olho de peixe \textit{full-frame} com ângulos de visão diagonal de 180º, 122º e 103,7º. 

Desta forma, como a aplicação da lente é no meio vinicula e num plano de imagem de uma lente olho de peixe circular maior parte da imagem é céu o uso da lente olho de peixe \textit{full-frame} é vantajoso para obter maior informação na horizontal do campo de visão e melhor na vertical.  

\begin{figure}[h!] %colocar figura a seguir ao texto anterior
	\begin{center}
		\leavevmode		
		\includegraphics[width=0.6\textwidth]{circularfullframe.jpg}
		\caption{Diferença de ângulos de visão de lentes olho de peixe.}
		\label{fig:circularfullframe}
	\end{center}
\end{figure}


A lente especificamente escolhida é a representada na figura ~\ref{fig:lentfisheye}, com a seguintes especificações: 5 megapixel , angulo de abertura de 160 graus (câmaras têm tipicamente 72 graus), resolução 1080p e com suporte para LED infravermelho para visão noturna.

\begin{figure}[h!]%colocar figura a seguir ao texto anterior
	\begin{center}
		\leavevmode		
		\includegraphics[width=0.4\textwidth]{lentfisheye}
		\caption{Câmara com lente olho de peixe e suporte para visão noturna.}
		\label{fig:lentfisheye}
	\end{center}
\end{figure}
