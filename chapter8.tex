\chapter{Conclusões e Trabalho Futuro} \label{chap:conclF}

Neste capítulo serão apresentadas as conclusões finais sobre os resultados da dissertação e melhoria do sistema desenvolvido.

\section{Conclusões}


Em forma de conclusão a presente dissertação demonstrou a importância da Odometria Visual para as encostas da vinha.
De salientar, logo em primeira vista que, a Odometria Visual no teste  estático apresenta vantagens  que se tornam significativa para a localização. Exemplo disso  é que quando o robô patina, a Odometria das rodas obtém valores errados, o que não se verifica na Odometria Visual.
Alem disso, quando o movimento não é estático os resultado são idênticos, ou seja, o deslocamento é igual . Contudo, são visíveis erros pelo facto da não existência de uma escala, que varia consoante a velocidade de deslocamento para possíveis comparações.
Em relação às rotações estas têm resultados positivos obtendo erros mínimos comparados com a Odometria das rodas do robô. Salientar, apenas que o posicionamento do robô é afetado devido a movimentos da cabeça do robô que originam erros nas imagens.
Existe ainda,  outros tipos de erros que advém do contexto em que é realizado. Mas, em geral a Odometria Visual tem vantagens nas encostas devido aos erros obtidos na Odometria das rodas não afetarem a Odometria Visual.

\section{Trabalho Futuro}


De forma a colmatar erros existentes e visando a melhoria do projeto desenvolvido torna-se essencial definir o fator de escala para diferentes velocidades de forma a obter resultados de deslocamento em metros. Outro ponto a melhorar será a escolha dos pontos característicos de forma a obter pontos mais concentrados no meio envolvido, nas vinhas para melhor precisão dos cálculos de deslocamento. Por último seria útil o uso da Odometria Visual com a Odometria das rodas de forma a obter uma conjugação eficaz, de maneira a se obter o melhor resultado de ambas.