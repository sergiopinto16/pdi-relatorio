\chapter{Trabalho} \label{chap:sota}
\section{Interações com orientador}\label{chap:descricao}
Na fase Reordenação de Escolhas pelo Estudante, a 5 de Outubro, esclareci alguma dúvidas por email com o professor Armando Sousa sobre o tema da dissertação.
		
A primeira reunião, a 6 de Novembro, serviu para esclarecer quais os objetivos da dissertação e para o Orientador recomendar alguns temas a serem estudados pelo estudante, mais especificamente, estudo do \textit{ROS}, leitura de alguns documentos sobre trabalhos realizados no projeto \textit{RoMoVi}, contextualização com odometria visual e processamento/análise de imagens.

A 13 de Novembro realizou-se outra reunião para esclarecimento de dúvidas sobre os tópicos: utilizar imagem mono ou stereo , especificações necessárias na lente olho de peixe, \textit{ROS}.

Por ultimo existiu contacto por email, para receber feedback acerca deste documento.
	

\section{Trabalho realizado}\label{sec:proximas etapas}

Inicialmente foi feita uma leitura de todos os documentos aconselhados pelo Orientador. Desta forma foram surgindo duvidas das quais resultaram pesquisas na internet e outras foram esclarecidas pelo Orientador. Mais tarde foi realizado um estudo geral do mercado e quais os componentes necessário a utilizar na dissertação.

Em paralelo foi inicializada a contextualização com o \textit{ROS}, realizando a sua instalação e primeiros tutoriais.
