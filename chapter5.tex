\chapter{Plano de Trabalho} \label{chap:concl}

Neste capítulo, serão apresentadas as principais tarefas de desenvolvimento do projeto, a metodologia adotada e as ferramentas utilizadas. Na Figura ~\ref{fig:gantt} podemos ver um diagrama de \textit{Gantt} referente às principais tarefas a realizar:

\begin{figure}[h!]
	\begin{ganttchart}[y unit title=0.4cm,
	y unit chart=0.5cm,
	vgrid,hgrid,
	title height=1,
	bar/.style={draw,fill=cyan},
	bar incomplete/.append style={fill=yellow!50},
	bar height=0.7]{1}{24}
	\gantttitle{Fevereiro}{4}
	\gantttitle{Março}{4}
	\gantttitle{Abril}{4}
	\gantttitle{Maio}{4}
	\gantttitle{Junho}{4}
	\gantttitle{Julho}{4} \\
	\ganttbar{Fase 1}{1}{3} \\
	\ganttbar{Fase 2}{4}{6} \\
	\ganttbar{Fase 3}{7}{12} \\
	\ganttbar{Fase 4}{13}{18} \\
	\ganttbar{Fase 5}{19}{21} \\
	\ganttbar{Conclus\~ao}{22}{24} \\
	% rela\c c\~oes
	\ganttlink{elem0}{elem1}
	\ganttlink{elem1}{elem2}
	\ganttlink{elem2}{elem3}
	\ganttlink{elem3}{elem4}
	\ganttlink{elem4}{elem5}
	
\end{ganttchart}

	\caption{Diagrama de \textit{Gantt} com o planeamento do tempo para as principais tarefas.}
	\label{fig:gantt}
\end{figure}


\begin{enumerate}
	\item Fase 1: Obter uma comunicação entre a câmara e o raspberry pi através do uso de ROS (do inglês, \textit{Robot Operating System}), de forma a não limitar o uso da câmara nem do raspberry pi.
	\item Fase 2: Implementar os métodos analisados nos capítulos anteriores.
	\item Fase 3: Calcular a distância entre as imagens através do conhecimento da formação de uma imagem e realizar testes.
	\item Fase 4: Implementar os métodos em diferente câmaras , tais como FLIR (do inglês, \textit{Forward Looking Infra-Red} ) devido, ao facto, de o projeto RoMoVi já ter incorporado uma para deteção de troncos.
	\item Fase 5: Testes finais.
	\item Fase 6: Escrita da tese.
\end{enumerate}

A metodologia adotada consiste em implementar os métodos já existentes e realizar ajustes para o caso das câmaras com lentes de olho de peixe. 

Em termos de ferramentas a utilizar, será necessário a utilização de ROS (do inglês, \textit{Robot Operating System}), este fornece bibliotecas e ferramentas para ajudar os desenvolvedores de software a criar aplicações para robôs. A implementação deste software será em C/C++ desenvolvidas usando um \textit{IDE} sendo o eclipse o escolhido. Além disto, é necessário o uso de \textit{OpenCV}, uma biblioteca multiplataforma de forma a desenvolver os algoritmos de deteção , associação de pontos característicos e estimação do movimento.

Por último, a escrita é realizada na ferramenta \textit{Latex} , ideal para desenvolver documentos científicos devido à sua alta qualidade tipográfica.